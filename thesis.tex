% !TeX encoding = UTF-8
% !TeX program = pdflatex
% !TeX spellcheck = it_IT

\documentclass[Lau,binding=0.6cm,noexaminfo=true]{sapthesis}

\usepackage{microtype}
\usepackage[italian]{babel}
\usepackage[utf8]{inputenx}

\usepackage{hyperref}
\hypersetup{pdftitle={Analisi delle problematiche di sicurezza del protocollo MQTT},pdfauthor={Edoardo Di Paolo}}

% Remove in a normal thesis
\usepackage{lipsum}
\usepackage{curve2e}
\usepackage{setspace}
\onehalfspacing

\definecolor{gray}{gray}{0.4}
\newcommand{\bs}{\textbackslash}

% Commands for the titlepage
\title{Analisi delle problematiche di sicurezza del protocollo MQTT}
\author{Edoardo Di Paolo}
\IDnumber{1728334}
\course{Informatica}
\courseorganizer{Facoltà di Ingegneria dell'informazione, informatica e statistica}
\AcademicYear{2019/2020}
\copyyear{2020}
\advisor{Prof. Angelo Spognardi}
%\advisor{Dr. Nome Cognome}
%\coadvisor{Dr. Nome Cognome}
\authoremail{dipaolo.1728334@studenti.uniroma1.it}

%\examdate{ }
%\examiner{ }
%\examiner{Prof. Nome Cognome}
%\examiner{Dr. Nome Cognome}
%\versiondate{\today}




\begin{document}

\frontmatter

\maketitle

\dedication{Dediche.\\}

%\begin{acknowledgments}
%Ringraziamenti
%\end{acknowledgments}


%\begin{abstract}
%Introduzione
%\end{abstract}


\tableofcontents

% Do not use the starred version of the chapter command!



\mainmatter
\chapter{Introduzione}

\begin{large}
Negli ultimi decenni il numero di dispositivi collegati ad Internet è cresciuto esponenzialmente. Ormai, nel 2020, non si hanno più solamente computer o cellulari in rete ma anche elettrodomestici, macchine industriali e strumenti medici, tutti dispositivi che qualche anno fa erano offline. Tutto ciò fa riferimento all'IoT, l'\textit{Internet of Things}. \\
Parallelamente allo sviluppo di queste nuove tecnologie, sono stati studiati nuovi protocolli affinché i dispositivi potessero essere utilizzati in maniera efficiente. Ad esempio ci sono alcuni sensori i quali permettono di misurare temperatura ed umidità di una stanza e funzionano attraverso l'uso di una semplice pila; dunque è necessario andare a ridurre il costo energetico della connessione così da aumentare la durata di utilizzo del dispositivo. \\
Alcuni esempi di protocolli possono essere: MQTT, CoAP, AMQP e WebSocket. Ovviamente tutti i protocolli hanno la possibilità di essere integrati con TLS (\textit{Transport Layer Security}) così da poter garantire una maggiore sicurezza nello scambio dei dati; questo, però, potrebbe andare a gravare sui consumi del dispositivo poiché dovrebbero essere effettuati più calcoli affinché avvenga il trasporto dei dati. \\

Con l'aumento di questi nuovi dispositivi sono aumentate anche le possibili minacce relative all'IoT. Un esempio è il malware Mirai che, nel 2016, ha infettato milioni di dispositivi rendendoli parte di una botnet la quale, successivamente, ha attaccato il fornitore di servizi DNS Dyn così da rendere inaccessibili milioni di siti web.
\end{large}




% bibliography
%\cleardoublepage
%\phantomsection
\bibliographystyle{sapthesis} % BibTeX style
\bibliography{bibliography} % BibTeX database without .bib extension

\end{document}
